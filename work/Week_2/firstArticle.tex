\documentclass[letterpaper,notitlepage,11pt]{article}
\usepackage[margin=2cm]{geometry}
\usepackage{color}
\begin{document}
\title{LaTeX Description}
\author{Zakir Alibhai}
\date{\today}
\maketitle
\begin{abstract}
TeX is a typesetting computer program created by Donald Knuth, originally for his magnum opus, The Art of Computer Programming. It takes a "plain" text file and converts it into a high-quality document for printing or on-screen viewing. LaTeX is a macro system built on top of TeX that aims to simplify its use and automate many common formatting tasks. It is the de-facto standard for academic journals and books in several fields, such as mathematics and physics, and provides some of the best typography free software has to offer.
\end{abstract}
\section{Introduction}
LaTeX is a \\ computer program used for \break making articles, books and math formulas look good. LaTeX is \hfil\break well-suited for expressing mathematical formulas on electronic devices in a more human readable format, by showing them in a way similar to how they would be written by hand.

LaTeX is used \hfil\break for making mathematical formulas for some articles on Wikipedia, in addition to being used within academic circles.

The writer types their article into a plain text document. A plain text document cannot have styled text, like bold or italic. When the writer wants to write styled text, they use special LaTeX commands that start with a backslash. For example, the command for bold text is \textbf{This text is bold}.

After the writer is finished writing the article, they tell LaTeX to read the document. After LaTeX is done, LaTeX makes a file that can be printed. The command \textbf{This text is bold} would print as This text is bold.

LaTeX was first made in the early 1980s by Leslie Lamport at SRI International, who published its first manual in 1986.[1] The current version is LaTeX2e (styled \LaTeXe), which has been active since 1994.[2]

\end{document}