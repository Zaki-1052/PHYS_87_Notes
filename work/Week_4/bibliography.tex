\documentclass[letterpaper,notitlepage,11pt]{article}
\begin{document}
\title{LaTeX Description}
\author{Zakir Alibhai}
\date{\today}
\maketitle
\begin{abstract}
TeX is a typesetting computer program created by Donald Knuth, originally for his magnum opus, The Art of Computer Programming. It takes a "plain" text file and converts it into a high-quality document for printing or on-screen viewing.
\end{abstract}
\section{Introduction}
LaTeX is a computer program used for making articles, books and math formulas look good. LaTeX is well-suited for expressing mathematical formulas on electronic devices\cite{grinstein} in a more human readable format, by showing them in a way similar to how they would be written by hand.

LaTeX is used or making mathematical formulas for some articles on Wikipedia, in addition to being used within academic circles\cite{soil}.

The writer types their article into a plain text document. A plain text document cannot have styled text, like bold or italic. When the writer wants to write styled text, they use special LaTeX commands that start with a backslash. For example, the command for bold text is \textbf{This text is bold}.

\bibliographystyle{abbrv}
\bibliography{bibliography}

\end{document}